\documentclass[../main.tex]{subfiles}

\graphicspath{{\subfix{../imgs/}}}

\begin{document}

\section{Abstract}

In recent years, there has been a growing interest in developing systems that can continuously
monitor biomedical signals for healthcare and lifestyle applications. This interest has been
driven by significant advancements in technologies used to measure signals like EEG, ECG,
and EMG. \vspace*{10pt}

For instance, EEG is a method used for diagnosing epilepsy and is also commonly used in
Brain-Computer Interfaces (BCI). However, EEG signals are quite weak, with very low strength
and slow frequencies (i.e., $10-100 \si{\mu V}$ and $0.5-50\si{Hz}$). This makes EEG signals easily influenced
by external noise and unwanted artifacts, such as those caused by eye movements, heart
activity, and muscle contractions. Additionally, it's important for these monitoring systems to
have low power consumption to ensure longer battery life for continuous signal recording. \vspace*{10pt}

This project aims to design and implement a low noise amplifier and a low-pass filter for an
EEG analog front-end in a 65nm CMOS technology with a primary emphasis on minimizing
input-referred noise and power consumption. For EEG recordings, these two blocks are critical
components of the analog front-end, which must align with the demands of bandwidth, noise,
and power consumption. \vspace*{10pt}

\noindent \textbf{Keywords: CMOS, EEG, LNA, Analog Front-End, Low Power, Low Noise, 65nm Technology.}

\end{document}